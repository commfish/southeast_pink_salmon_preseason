\documentclass[]{article}
\usepackage{lmodern}
\usepackage{amssymb,amsmath}
\usepackage{ifxetex,ifluatex}
\usepackage{fixltx2e} % provides \textsubscript
\ifnum 0\ifxetex 1\fi\ifluatex 1\fi=0 % if pdftex
  \usepackage[T1]{fontenc}
  \usepackage[utf8]{inputenc}
\else % if luatex or xelatex
  \ifxetex
    \usepackage{mathspec}
  \else
    \usepackage{fontspec}
  \fi
  \defaultfontfeatures{Ligatures=TeX,Scale=MatchLowercase}
\fi
% use upquote if available, for straight quotes in verbatim environments
\IfFileExists{upquote.sty}{\usepackage{upquote}}{}
% use microtype if available
\IfFileExists{microtype.sty}{%
\usepackage[]{microtype}
\UseMicrotypeSet[protrusion]{basicmath} % disable protrusion for tt fonts
}{}
\PassOptionsToPackage{hyphens}{url} % url is loaded by hyperref
\usepackage[unicode=true]{hyperref}
\hypersetup{
            pdfborder={0 0 0},
            breaklinks=true}
\urlstyle{same}  % don't use monospace font for urls
\usepackage[margin=1in]{geometry}
\usepackage{longtable,booktabs}
% Fix footnotes in tables (requires footnote package)
\IfFileExists{footnote.sty}{\usepackage{footnote}\makesavenoteenv{long table}}{}
\usepackage{graphicx,grffile}
\makeatletter
\def\maxwidth{\ifdim\Gin@nat@width>\linewidth\linewidth\else\Gin@nat@width\fi}
\def\maxheight{\ifdim\Gin@nat@height>\textheight\textheight\else\Gin@nat@height\fi}
\makeatother
% Scale images if necessary, so that they will not overflow the page
% margins by default, and it is still possible to overwrite the defaults
% using explicit options in \includegraphics[width, height, ...]{}
\setkeys{Gin}{width=\maxwidth,height=\maxheight,keepaspectratio}
\IfFileExists{parskip.sty}{%
\usepackage{parskip}
}{% else
\setlength{\parindent}{0pt}
\setlength{\parskip}{6pt plus 2pt minus 1pt}
}
\setlength{\emergencystretch}{3em}  % prevent overfull lines
\providecommand{\tightlist}{%
  \setlength{\itemsep}{0pt}\setlength{\parskip}{0pt}}
\setcounter{secnumdepth}{0}
% Redefines (sub)paragraphs to behave more like sections
\ifx\paragraph\undefined\else
\let\oldparagraph\paragraph
\renewcommand{\paragraph}[1]{\oldparagraph{#1}\mbox{}}
\fi
\ifx\subparagraph\undefined\else
\let\oldsubparagraph\subparagraph
\renewcommand{\subparagraph}[1]{\oldsubparagraph{#1}\mbox{}}
\fi

% set default figure placement to htbp
\makeatletter
\def\fps@figure{htbp}
\makeatother


\author{}
\date{\vspace{-2.5em}}

\begin{document}

title: ``Vessel Calibration Coefficient Discussion'' author: ``Sara
Miller, Rich Brenner, Jim Murphy'' date: ``November 2, 2020--draft''

output: bookdown:: word\_document: fig\_caption: yes toc: yes
header-includes: \usepackage{float}

\section{Calibration Coefficient
Discussion}\label{calibration-coefficient-discussion}

\subsection{Background}\label{background}

Excerpted from Wertheimer et al. 2010:

``From 1997 to 2007, SECM used the NOAA ship \emph{John N. Cobb} to
accrue an 11 year time series of catches with a Nordic 264 rope trawl
fished at the surface\ldots{} (Orsi et al. 2000, 2008)\ldots{} In 2007,
in anticipation of the decommissioning of the \emph{John N. Cobb}, the
\emph{Medeia} and the \emph{John N. Cobb} fished synoptically for 28
pairs of trawl hauls to develop calibration factors in the event of
differential catch rates between the two vessels (Wertheimer et al.
2008). In 2008, the \emph{Medeia} fished synoptically with the chartered
research vessel \emph{Steller} to determine relative fishing efficiency
so that \emph{Steller} catches could then be compared and calibrated to
the SECM data series from the \emph{John N. Cobb} (Wertheimer et al.
2009). In 2009, the commercial trawler \emph{Chellissa} was chartered to
fish the SECM transects in the northern and southern regions of
Southeast Alaska. The \emph{Medeia} was again fished synoptically in the
northern region transects to determine relative fishing efficiency
(Table 1).''

\pagebreak

\begin{longtable}[]{@{}lrrrrrr@{}}
\caption{Estimated fishing power coefficients for juvenile salmon
catches by the different vessels used during the Southeast Alaska
Coastal Monitoring survey (Wertheimer et al. 2008, 2009, and
2010).\emph{Chellissa:Cobb} was calculated from pink salmon estimates
for \emph{Chellisa:Medeia} and \emph{Medeia:Cobb}. \emph{Mixed
Chellissa:Cobb} was a mixture of species estimates for
\emph{Chellisa:Medeia} and a pooled species estimate for
\emph{Medeia:Cobb}. One of the primary trawl vessels, F/V
\emph{Northwest Explorer}, has not been calibrated and it is assumed to
have the same fishing power as the \emph{Chellissa}.}\tabularnewline
\toprule
Species & Medeia:Cobb & Chellissa:Cobb & Steller:Cobb & Medeia:Steller &
Chellissa:Medeia & mixed Chellissa:Cobb\tabularnewline
\midrule
\endfirsthead
\toprule
Species & Medeia:Cobb & Chellissa:Cobb & Steller:Cobb & Medeia:Steller &
Chellissa:Medeia & mixed Chellissa:Cobb\tabularnewline
\midrule
\endhead
Pink & 1.13 & 1.44 & 0.96 & 1.18 & 1.27 & 1.51\tabularnewline
Chum & 1.21 & 1.44 & 1.16 & 1.04 & 1.19 & 1.42\tabularnewline
Sockeye & 1.19 & 1.18 & 1.05 & 1.13 & 0.99 & 1.18\tabularnewline
Coho & 1.26 & 1.32 & 0.85 & 1.48 & 1.05 & 1.25\tabularnewline
Total Salmon & 1.19 & 1.36 & 1.05 & 1.13 & 1.14 & 1.36\tabularnewline
\bottomrule
\end{longtable}

For the 2021 SEAK pink salmon forecast, there was a discussion as to
which vessel calibration coefficient to use going forward. Using the
four potential vessel calibration coefficients (pink\_cal\_mixspecies,
pink\_cal\_mixpool, pink\_cal\_species, pink\_cal\_pool; Table 2), the
corresponding index of juvenile abundance was slightly different (i.e.,
CPUE; standardized pink salmon catch based on a 20 minute trawl set by
year; Table 3). Performance metrics (Akaike Information Criterion
corrected for small sample sizes; AICc values; Burnham and Anderson
2004; mean and median absolute percentage error (MAPE, MEAPE); mean
absolute scaled error (MASE) (Hyndman and Kohler 2006)) were used to
evaluate forecast accuracy of alternative vessel calibration
coefficients (Table 4) using the same model. Statistical analyses were
performed with the R Project for Statistical computing version 3.6.3 (R
Core Team 2020). The model used for the comparison of the vessel
calibration coefficient model was:

\[E(y) = \alpha + \beta_1{X_1} + \beta_2{X_2},\]

where \({X_1}\) is CPUE, juvenile pink salmon abundance index based on
the different vessel calibration coefficients, and \({X_2}\) is the
average temperature in Icy Strait in May, June, and July at eight
stations in Icy Strait. CPUE data are log-transformed catches that are
standardized to an effort of a 20 minute trawl set. The four potential
vessel calibration coefficients are defined as:

pink\_cal\_mixspecies is a mixture of pink-specific
(\emph{Chellisa:Medeia}) and pooled-species (\emph{Medeia:Cobb})
coefficients for the \emph{Chellissa} and the \emph{N/W Explorer}, and a
pink salmon coefficient for the \emph{Medeia}. This ends up with a time
series that is based on coefficients that vary with species, but are
partially derived from a mixture of species-specific and pooled-species
coefficients.

pink\_cal\_mixpool is a mixture of pink-specific
(\emph{Chellissa:Medeia}) and pooled-species (\emph{Medeia:Cobb})
coefficients for the \emph{Chellissa} and the \emph{N/W Explorer}, and a
pooled-species coefficient for the \emph{Medeia}.

pink\_cal\_species\ldots{}

pink\_cal\_pool\ldots{}

\begin{longtable}[]{@{}lrrrr@{}}
\caption{Calibration coefficients used to convert vessel-specific
catches to \emph{Cobb} units. Direct calibrations with the \emph{Cobb}
are estimated for the \emph{Steller} and \emph{Medeia}, therefore mixed
coefficients are only applied to the \emph{Chellissa} and \emph{NW
Explorer}. Species-specific or pooled-species coefficients could be used
as the mixed coefficients for the \emph{Steller} and
\emph{Medeia}.}\tabularnewline
\toprule
Vessel & pink\_cal\_mixspecies & pink\_cal\_mixpool & pink\_cal\_species
& pink\_cal\_pool\tabularnewline
\midrule
\endfirsthead
\toprule
Vessel & pink\_cal\_mixspecies & pink\_cal\_mixpool & pink\_cal\_species
& pink\_cal\_pool\tabularnewline
\midrule
\endhead
Cobb & 1.00 & 1.00 & 1.00 & 1.00\tabularnewline
Chellissa & 0.66 & 0.66 & 0.70 & 0.74\tabularnewline
NW Explorer & 0.66 & 0.66 & 0.70 & 0.74\tabularnewline
Steller & 1.04 & 0.95 & 1.04 & 0.95\tabularnewline
Medeia & 0.88 & 0.84 & 0.88 & 0.84\tabularnewline
\bottomrule
\end{longtable}

\begin{longtable}[]{@{}lrrrr@{}}
\caption{The data for the variable CPUE (index of juvenile pink salmon
abundance based on log-transformed catches that are standardized to an
effort of a 20 minute trawl set) using different vessel calibration
coefficients.}\tabularnewline
\toprule
year & pink\_cal\_mixspecies & pink\_cal\_mixpool & pink\_cal\_species &
pink\_cal\_pool\tabularnewline
\midrule
\endfirsthead
\toprule
year & pink\_cal\_mixspecies & pink\_cal\_mixpool & pink\_cal\_species &
pink\_cal\_pool\tabularnewline
\midrule
\endhead
1997 & 2.477744 & 2.477744 & 2.4777444 & 2.4777444\tabularnewline
1998 & 5.622380 & 5.622380 & 5.6223800 & 5.6223800\tabularnewline
1999 & 1.597723 & 1.597723 & 1.5977233 & 1.5977233\tabularnewline
2000 & 3.729985 & 3.729985 & 3.7299847 & 3.7299847\tabularnewline
2001 & 2.868826 & 2.868826 & 2.8688260 & 2.8688260\tabularnewline
2002 & 2.784664 & 2.784664 & 2.7846641 & 2.7846641\tabularnewline
2003 & 3.077820 & 3.077820 & 3.0778204 & 3.0778204\tabularnewline
2004 & 3.899407 & 3.899407 & 3.8994067 & 3.8994067\tabularnewline
2005 & 2.040345 & 2.040345 & 2.0403454 & 2.0403454\tabularnewline
2006 & 2.572781 & 2.572781 & 2.5727807 & 2.5727807\tabularnewline
2007 & 1.167639 & 1.167639 & 1.1676386 & 1.1676386\tabularnewline
2008 & 2.555111 & 2.323473 & 2.5551107 & 2.3234731\tabularnewline
2009 & 2.094192 & 2.094192 & 2.2053878 & 2.3330031\tabularnewline
2010 & 3.687782 & 3.687782 & 3.8835930 & 4.1083181\tabularnewline
2011 & 1.305918 & 1.305918 & 1.3752584 & 1.4548381\tabularnewline
2012 & 3.161036 & 3.161036 & 3.3288788 & 3.5215052\tabularnewline
2013 & 1.923429 & 1.923429 & 2.0255583 & 2.1427677\tabularnewline
2014 & 3.426619 & 3.426619 & 3.6085629 & 3.8173733\tabularnewline
2015 & 2.201588 & 2.201588 & 2.3184867 & 2.4526465\tabularnewline
2016 & 3.905705 & 3.905705 & 4.1130877 & 4.3510925\tabularnewline
2017 & 0.310436 & 0.310436 & 0.3269194 & 0.3458366\tabularnewline
2018 & 1.233765 & 1.171558 & 1.2337651 & 1.1715584\tabularnewline
2019 & 1.202606 & 1.141971 & 1.2026065 & 1.1419709\tabularnewline
2020 & 2.261529 & 2.147502 & 2.2615289 & 2.1475023\tabularnewline
average & 2.546210 & 2.526689 & 2.5949191 & 2.6313000\tabularnewline
\bottomrule
\end{longtable}

\subsection{Conclusion}\label{conclusion}

Based on \emph{x,y,and z???}, the discussion was limited to the
pink\_cal\_species and pink\_cal\_pool vessel calibration coefficients.
There is a bit more statistical support for the pink\_cal\_pool
coefficient. As the pool coefficients are currently used for sockeye,
coho and Chinook salmon catches, the pink\_cal\_pool vessel calibration
coefficient will be used moving forward for calculating the juvenile
pink salmon abundance index.

\begin{longtable}[]{@{}lrrrrrrrrl@{}}
\caption{Comparison of the performance metrics for a model based on
calculating a time-series of juvenile pink salmon abundance using
different vessel calibration coefficients.}\tabularnewline
\toprule
model & AdjR2 & AICc & MAPE & MEAPE & MASE & forecast & lower\_80 &
upper\_80 & index\tabularnewline
\midrule
\endfirsthead
\toprule
model & AdjR2 & AICc & MAPE & MEAPE & MASE & forecast & lower\_80 &
upper\_80 & index\tabularnewline
\midrule
\endhead
CPUE+ISTI & 0.811 & 16 & 0.079 & 0.061 & 0.262 & 31.3 & 20.9 & 46.9 &
pink\_cal\_mixspecies\tabularnewline
CPUE+ISTI & 0.820 & 15 & 0.077 & 0.060 & 0.256 & 29.8 & 20.1 & 44.1 &
pink\_cal\_mixpool\tabularnewline
CPUE+ISTI & 0.820 & 15 & 0.077 & 0.070 & 0.257 & 30.6 & 20.6 & 45.3 &
pink\_cal\_species\tabularnewline
CPUE+ISTI & 0.830 & 14 & 0.074 & 0.060 & 0.249 & 28.5 & 19.4 & 41.7 &
pink\_cal\_pool\tabularnewline
\bottomrule
\end{longtable}

\section{References}\label{references}

Burnham, K. P., and D. R. Anderson. 2004. Multimodel inference:
Understanding AIC and BIC in model selection. Sociological Methods and
Research 33: 261-304.

Hyndman, R. J. and A. B. Koehler. 2006. Another look at measures of
forecast accuracy. International Journal of Forecasting 22: 679-688.

Orsi, J. A., M. V. Sturdevant, J. M. Murphy, D. G. Mortensen, and B. L.
Wing. 2000. Seasonal habitat use and early marine ecology of juvenile
Pacific salmon in southeastern Alaska. NPAFC Bull. 2:111-122.

Orsi, J. A., E. A. Fergusson, M. V. Sturdevant, B. L. Wing, A. C.
Wertheimer, and W. R. Heard. 2008. Annual survey of juvenile salmon and
ecologically related species and environmental factors in the marine
waters of southeastern Alaska, May--August 2007. NPAFC Doc. 1110, 82 pp.

R Core Team. 2020. R: A language and environment for statistical
computing. R Foundation for Statistical Computing, Vienna, Austria. URL:
\url{http://www.r-project.org/index.html}

Wertheimer, A. C., J. A. Orsi, E. A. Fergusson, and M. V. Sturdevant.
2008. Paired comparisons of juvenile salmon catches between two research
vessels fishing Nordic 264 surface trawls in southeastern Alaska, July
2007. NPAFC Doc. 1112.,17 p.

Wertheimer, A. C., J. A. Orsi, E. A. Fergusson, and M. V. Sturdevant.
2009. Calibration of Juvenile Salmon Catches using Paired Comparisons
between Two Research Vessels Fishing Nordic 264 Surface Trawls in
Southeastern Alaska, July 2008. NPAFC Doc. 1180. 18 pp.

Wertheimer, A. C., J. A. Orsi, E. A. Fergusson, and M. V. Sturdevant.
2010. Calibration of Juvenile Salmon Catches using Paired Comparisons
between Two Research Vessels Fishing Nordic 264 Surface Trawls in
Southeast Alaska, July 2009. NPAFC Doc. 1177. 19 pp. (Available at
\url{http://www.npafc.org}).

\end{document}
