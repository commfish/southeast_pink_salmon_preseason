% Options for packages loaded elsewhere
% Options for packages loaded elsewhere
\PassOptionsToPackage{unicode}{hyperref}
\PassOptionsToPackage{hyphens}{url}
\PassOptionsToPackage{dvipsnames,svgnames,x11names}{xcolor}
%
\documentclass[
  letterpaper,
  DIV=11,
  numbers=noendperiod]{scrartcl}
\usepackage{xcolor}
\usepackage{amsmath,amssymb}
\setcounter{secnumdepth}{-\maxdimen} % remove section numbering
\usepackage{iftex}
\ifPDFTeX
  \usepackage[T1]{fontenc}
  \usepackage[utf8]{inputenc}
  \usepackage{textcomp} % provide euro and other symbols
\else % if luatex or xetex
  \usepackage{unicode-math} % this also loads fontspec
  \defaultfontfeatures{Scale=MatchLowercase}
  \defaultfontfeatures[\rmfamily]{Ligatures=TeX,Scale=1}
\fi
\usepackage{lmodern}
\ifPDFTeX\else
  % xetex/luatex font selection
\fi
% Use upquote if available, for straight quotes in verbatim environments
\IfFileExists{upquote.sty}{\usepackage{upquote}}{}
\IfFileExists{microtype.sty}{% use microtype if available
  \usepackage[]{microtype}
  \UseMicrotypeSet[protrusion]{basicmath} % disable protrusion for tt fonts
}{}
\makeatletter
\@ifundefined{KOMAClassName}{% if non-KOMA class
  \IfFileExists{parskip.sty}{%
    \usepackage{parskip}
  }{% else
    \setlength{\parindent}{0pt}
    \setlength{\parskip}{6pt plus 2pt minus 1pt}}
}{% if KOMA class
  \KOMAoptions{parskip=half}}
\makeatother
% Make \paragraph and \subparagraph free-standing
\makeatletter
\ifx\paragraph\undefined\else
  \let\oldparagraph\paragraph
  \renewcommand{\paragraph}{
    \@ifstar
      \xxxParagraphStar
      \xxxParagraphNoStar
  }
  \newcommand{\xxxParagraphStar}[1]{\oldparagraph*{#1}\mbox{}}
  \newcommand{\xxxParagraphNoStar}[1]{\oldparagraph{#1}\mbox{}}
\fi
\ifx\subparagraph\undefined\else
  \let\oldsubparagraph\subparagraph
  \renewcommand{\subparagraph}{
    \@ifstar
      \xxxSubParagraphStar
      \xxxSubParagraphNoStar
  }
  \newcommand{\xxxSubParagraphStar}[1]{\oldsubparagraph*{#1}\mbox{}}
  \newcommand{\xxxSubParagraphNoStar}[1]{\oldsubparagraph{#1}\mbox{}}
\fi
\makeatother


\usepackage{longtable,booktabs,array}
\usepackage{calc} % for calculating minipage widths
% Correct order of tables after \paragraph or \subparagraph
\usepackage{etoolbox}
\makeatletter
\patchcmd\longtable{\par}{\if@noskipsec\mbox{}\fi\par}{}{}
\makeatother
% Allow footnotes in longtable head/foot
\IfFileExists{footnotehyper.sty}{\usepackage{footnotehyper}}{\usepackage{footnote}}
\makesavenoteenv{longtable}
\usepackage{graphicx}
\makeatletter
\newsavebox\pandoc@box
\newcommand*\pandocbounded[1]{% scales image to fit in text height/width
  \sbox\pandoc@box{#1}%
  \Gscale@div\@tempa{\textheight}{\dimexpr\ht\pandoc@box+\dp\pandoc@box\relax}%
  \Gscale@div\@tempb{\linewidth}{\wd\pandoc@box}%
  \ifdim\@tempb\p@<\@tempa\p@\let\@tempa\@tempb\fi% select the smaller of both
  \ifdim\@tempa\p@<\p@\scalebox{\@tempa}{\usebox\pandoc@box}%
  \else\usebox{\pandoc@box}%
  \fi%
}
% Set default figure placement to htbp
\def\fps@figure{htbp}
\makeatother





\setlength{\emergencystretch}{3em} % prevent overfull lines

\providecommand{\tightlist}{%
  \setlength{\itemsep}{0pt}\setlength{\parskip}{0pt}}



 


\KOMAoption{captions}{tableheading}
\makeatletter
\@ifpackageloaded{caption}{}{\usepackage{caption}}
\AtBeginDocument{%
\ifdefined\contentsname
  \renewcommand*\contentsname{Table of contents}
\else
  \newcommand\contentsname{Table of contents}
\fi
\ifdefined\listfigurename
  \renewcommand*\listfigurename{List of Figures}
\else
  \newcommand\listfigurename{List of Figures}
\fi
\ifdefined\listtablename
  \renewcommand*\listtablename{List of Tables}
\else
  \newcommand\listtablename{List of Tables}
\fi
\ifdefined\figurename
  \renewcommand*\figurename{Figure}
\else
  \newcommand\figurename{Figure}
\fi
\ifdefined\tablename
  \renewcommand*\tablename{Table}
\else
  \newcommand\tablename{Table}
\fi
}
\@ifpackageloaded{float}{}{\usepackage{float}}
\floatstyle{ruled}
\@ifundefined{c@chapter}{\newfloat{codelisting}{h}{lop}}{\newfloat{codelisting}{h}{lop}[chapter]}
\floatname{codelisting}{Listing}
\newcommand*\listoflistings{\listof{codelisting}{List of Listings}}
\makeatother
\makeatletter
\makeatother
\makeatletter
\@ifpackageloaded{caption}{}{\usepackage{caption}}
\@ifpackageloaded{subcaption}{}{\usepackage{subcaption}}
\makeatother
\usepackage{bookmark}
\IfFileExists{xurl.sty}{\usepackage{xurl}}{} % add URL line breaks if available
\urlstyle{same}
\hypersetup{
  pdftitle={SEAK Pink Salmon Forecast},
  pdfauthor={Sara Miller},
  colorlinks=true,
  linkcolor={blue},
  filecolor={Maroon},
  citecolor={Blue},
  urlcolor={Blue},
  pdfcreator={LaTeX via pandoc}}


\title{SEAK Pink Salmon Forecast}
\author{Sara Miller}
\date{}
\begin{document}
\maketitle


\section{Objective}\label{objective}

To forecast the Southeast Alaska (SEAK) pink salmon commercial harvest
in 2026. This document is for guidance as to what was done for the
current forecast year. It is for internal use only.

\section{Executive Summary}\label{executive-summary}

Forecasts were developed using an approach originally described in
Wertheimer et al.~(2006), and modified in Orsi et al.~(2016) and Murphy
et al.~(2019), but assuming a log-normal error structure (Miller et
al.~2022). This approach is based on a multiple regression model with
the raw juvenile pink salmon catch-per-unit-effort (adj\_raw\_pink; a
proxy for abundance), a vessel factor to account for the survey vessels
through time, an odd and even year factor to account for potential odd
and even year cycles of abundance, and temperature data from the
Southeast Alaska Coastal Monitoring Survey (SECM; Piston et al.~2021;
ISTI20\_MJJ) or from satellite sea surface temperature (SST) data (Huang
et al.~2017). The adj\_raw\_pink variable is not the same as the CPUE
variable used in prior years that was adjusted by the pooled-species
vessel calibration coefficient for the Cobb and was the maximum average
from either June or July (whichever was higher). Instead, the CPUE term
used in the 2025 forecast, the adj\_raw\_pink variable, is the natural
logarithm of the maximum untransformed catch, adjusted to a 20 minute
haul, from either June or July. The Stellar and Chellissa vessels were
only used for one year each, 2008 and 2009, respectively, and so these
two years are not used in the assessment (Table 1).

There were 36 individual models considered:

\begin{itemize}
\item
  adj\_raw\_pink model with a vessel interaction, an odd/even year
  factor, and a vessel factor (m1);
\item
  adj\_raw\_pink model with a vessel interaction, an odd/even year
  factor, a vessel factor, and with temperature data from the the SECM
  survey (m2);
\item
  16 adj\_raw\_pink models with a vessel interaction, an odd/even year
  factor, a vessel factor, and with satellite SST data (m3-m18);
\item
  adjusted CPUE, and an odd/even year factor (m1a);
\item
  adjusted CPUE, an odd/even year factor, and with temperature data from
  the the SECM survey (m2a);
\item
  16 adjusted CPUE models with an odd/even year factor, and with
  satellite SST data (m3a-m18a);
\end{itemize}

The model performance metrics one-step ahead mean absolute percent error
(MAPE) for the last five years (MAPE5; forecast years 2021 through 2025)
was used to evaluate the forecast accuracy of the 36 individual models,
the AICc values were calculated for each model to prevent
over-parameterization of the model, and the adjusted R-squared values,
significant terms, and overall p-value of the model were used to the
determine fit. Based upon the performance metric the 5-year MAPE, the
AICc values, significant parameters in the models, and the adjusted
R-squared values, model xx (a model that included xx; Appendix B) was
the best performing model and the 2026-forecast using this model has a
point estimate of xx million fish (80\% prediction interval: xx to xx
million fish).

\section{Analysis}\label{analysis}

\subsection{Individual, multiple linear regression
models}\label{individual-multiple-linear-regression-models}

Biophysical variables based on data from Southeast Alaska were used to
forecast the harvest of adult pink salmon in Southeast Alaska, one year
in advance, using individual, multiple linear regression models (models
m1-m18 and models m1a--m18a). The two simplest regression model (model
m1 and model m1a) consisted of the predictor variable juvenile
adj\_raw\_pink \(({X_1})\) with a vessel factor interaction \(({X_V})\),
an odd/even year factor\(({X_B})\), and a vessel factor\(({X_V})\)
(model m1) and adjusted CPUE and an odd/even year factor\(({X_B})\)
(model m1a). The other 34 regression models (models m1-m18 and models
m1a-m18a) also included a temperature index \(({X_2})\). The general
model structure was

\[E(Y) = \hat{\beta_0} + \hat {\beta}_{1V}{X_V}{X_1} +\hat {\beta}_{2}{X_2}+\hat {\beta}_{B}{X_B}\].

The odd/even year factor adjusts the model intercept by
\(\hat{\beta_0}\) + \({\beta}_{B}\) as \({X_B}\) = 0 for even years (no
adjustment to intercept, defaults to model intercept) and \({X_B}\) = 1
for odd years (adjustment for odd years based on \({\beta}_{B}\)). The
vessel interaction adjusts the model slope. For example, during years
when the survey vessel was the Cobb, the slope is adjusted by the
\(\hat {\beta}_{1COBB}\) which is then multiplied by the adj\_raw\_pink
term in that year.

The temperature index for models m2d-m18d was either the SECM survey Icy
Strait temperature Index (ISTI20\_MJJ; Murphy et al.~2019) or one of the
16 satellite-derived SST data (Huang et al.~2017). Although the simplest
model did not contain a temperature variable, including temperature data
with CPUE has been shown to result in a substantial improvement in the
accuracy of model predictions (Murphy et al.~2019). The response
variable (\(Y\); Southeast Alaska adult pink salmon harvest in
millions), and the adj\_raw\_pink (CPUE) data were natural log
transformed in the model, but temperature data were not. The forecast
\((\hat{\textit {Y}_{i}})\), and 80\% prediction intervals (based on
output from program R; R Core Team 2023) from the 18 regression models
were exponentiated and bias-corrected (Miller 1984),

\[\hat{F_i} = \rm exp (\hat{\textit {Y}_{\textit i}} + \frac{{\sigma_i}^2}{2}),\tag{2}\]

where \({\hat {F_i}}\) is the preseason forecast (for each model \(i\))
in millions of fish, and \(\sigma_i\) is the variance (for each model
\(i\)).

\subsection{Performance metric: One-step ahead
MAPE}\label{performance-metric-one-step-ahead-mape}

The model summary results using the performance metric one-step ahead
MAPE are shown in Table 2; the smallest value is the preferred model
(Appendix C). The performance metric one-step ahead MAPE was calculated
as follows.

\begin{enumerate}
\def\labelenumi{\arabic{enumi}.}
\item
  Estimate the regression parameters at time \(t\)-1 from data up to
  time \(t\)-1.
\item
  Make a prediction of \({\hat{Y_t}}\) at time \(t\) based on the
  predictor variables at time \(t\) and the estimate of the regression
  parameters at time \(t\)-1 (i.e., the fitted regression equation).
\item
  Calculate the MAPE based on the prediction of \({\hat{Y_t}}\) at time
  \(t\) and the observed value of \({Y_t}\) at time \(t\),
\end{enumerate}

\[\text{MAPE} = |\frac{\rm exp{(\textit Y_{\textit t})} -\rm exp (\hat{\textit Y_{\textit t}} + \frac{{\sigma_t}^2}{2})}{\rm exp (\textit Y_{\textit t})}|.\tag{3}\]

\begin{enumerate}
\def\labelenumi{\arabic{enumi}.}
\setcounter{enumi}{3}
\tightlist
\item
  For each individual model, average the MAPEs calculated from the
  forecasts,
  \[\frac{1}{n} \sum_{t=1}^{n} |\frac{\rm exp{(\textit Y_{\textit t})} -\rm exp (\hat{\textit Y_{\textit t}} + \frac{{\sigma_t}^2}{2})}{\rm exp (\textit Y_{\textit t})}|,\tag{4}\]
  where \(n\) is the number of forecasts in the average (5 forecasts for
  the 5-year MAPE and 10 forecasts for the 10-year MAPE). For example,
  to calculate the five year one-step-ahead MAPE for model m1 for the
  2022 forecast, use data up through year 2016 (e.g., data up through
  year 2016 is \(t\) -1 and the forecast is for \(t\), or year 2017).
  Then, calculate a MAPE based on the 2017 forecast and the observed
  pink salmon harvest in 2017 using equation 3. Next, use data up
  through year 2017 (e.g., data up through year 2017 is \(t\) -1 and the
  forecast is for year 2018; \(t\)) and calculate a MAPE based on the
  2018 forecast and the observed pink salmon harvest in 2018 using
  equation 3. Repeat this process for each subsequent year through year
  2020 to forecast 2021. Finally, average the five MAPEs to calculate a
  five year one-step-ahead MAPE for model m1. As the results of the
  5-year MAPEs with or without the forecast bias adjustment have been
  similar (i.e., the model performance order did not change whether the
  five year one-step-ahead MAPE or the bias-corrected five year
  one-step-ahead MAPE was compared), for simplicity, the bias adjustment
  for the forecast was not used in the calculation of the five year
  one-step-ahead MAPE for model comparison.
\end{enumerate}

\subsection{Akaike Information Criterion corrected for small sample
sizes
(AICc)}\label{akaike-information-criterion-corrected-for-small-sample-sizes-aicc}

Hierarchical models were compared with the AICc criterion. The best fit
models, according to the AICc criterion, is one that explains the
greatest amount of variation with the fewest independent variables
(i.e., the most parsimonious; Table 2). The lower AICc values are
better, and the AICc criterion penalizes models that use more
parameters. Comparing the AICc values of two hierarchical models, a
\(\Delta_i \leq 2\) suggests that the two models are essentially the
same, and the most parsimonious model should be chosen (Burnham and
Anderson 2004). If the \(\Delta_i > 2\), the model with the lower AICc
value should be chosen.

\section{Results}\label{results}




\end{document}
